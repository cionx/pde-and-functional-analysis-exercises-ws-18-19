\section{}

For every~$n \in \Natural$ let~$e^{(n)} \in \ell^1$ be the sequence with~$e^{(n)}_n = 1$ and~$e^{(n)}_i = 0$ for all~$i \neq n$.





\subsection{}

We may regard~$A$ as a map~$A \colon \ell^\infty \to \Real^\Natural$, which is then linear.
We can then compute
\[
        \norm{a}_\infty
  =     \sup_{n \in \Natural} \abs{a_n}
  =     \sup_{n \in \Natural} \norm{A e^{(n)}}_1
  \leq  \norm{A} \,.
\]
We also have for every~$x \in \ell^1$ that
\[
        \norm{Ax}_1
  =     \sum_{n \in \Natural} \abs{a_n x_n}
  =     \sum_{n \in \Natural} \abs{a_n} \abs{x_n}
  \leq  \sum_{n \in \Natural} \norm{a}_\infty \abs{x_n}
  =     \norm{a}_\infty \sum_{n \in \Natural} \abs{x_n}
  =     \norm{a}_\infty \norm{x}_1  \,.
\]
This shows that~$\norm{A} = \norm{a}_\infty$.

Suppose now that~$a \in \ell^\infty$, so that~$\norm{A} = \norm{a}_\infty < \infty$.
We then find for every~$x \in \ell^1$ that~$\norm{Ax}_1 \leq \norm{A} \norm{x}_1 < \infty$, and hence that~$A$ restricts to a linear map~$\ell^1 \to \ell^1$.
We then have~$A \in \Lin(\ell^1)$ because~$\norm{A} < \infty$.

Suppose now that~$A \in \Lin(\ell^1)$.
Then in particular~$\norm{a}_\infty = \norm{A} < \infty$ and hence~$a \in \ell^1$.





\subsection{}

We have that
\begin{align*}
      \Null(A)
  &=  \{
        (x_n)_n \in \ell^1
      \suchthat
        (a_n x_n)_n = 0
      \}  \\
  &=  \{
        (x_n)_n \in \ell^1
      \suchthat
        \text{$x_n = 0$ for every~$n \in \Natural$ with~$a_n \neq 0$}
      \} \,.
\end{align*}
Hence~$\Null(A) = 0$ if and only if~$a_n \neq 0$ for every~$n \in \Natural$.
We now observe the following:

\begin{claim}
  \label{sequence in image}
  Let~$y = (y_n)_n \in \ell^1$ be a sequence with support~$S \defined \{n \in \Natural \suchthat y_n \neq 0\}$.
  If~$\inf_{n \in S} \abs{a_n} > 0$ then the sequence~$y$ is contained in the range~$A(\ell^1)$.
\end{claim}

\begin{proof}
  It holds in particular that~$a_n \neq 0$ for every~$n \in S$.
  The sequence~$x = (x_n)_n$ with
  \[
              x_n
    \defined  \begin{cases}
                y_n / a_n & \text{if~$n \in S$} \,, \\
                0         & \text{otherwise}  \,,
              \end{cases}
  \]
  is therefore {\welldef}.
  This sequence satisfies~$Ax = y$, and it is again contained in~$\ell^1$:
  The constant
  \[
              C
    \defined  \frac{1}{\inf_{n \in S} \abs{a_n}}
    =         \sup_{n \in S} \frac{1}{\abs{a_n}}
  \]
  is by assumption {\welldef}, and we have that
  \[
          \norm{x}_1
    =     \sum_{n \in \Natural} \abs{x_n}
    =     \sum_{n \in S} \frac{\abs{y_n}}{\abs{a_n}}
    \leq  C \sum_{n \in S} \abs{y_n}
    =     C \sum_{n \in \Natural} \abs{y_n}
    =     C \norm{y}_1 \,.
  \]
  It hence follows from~$y \in \ell^1$ that also~$x \in \ell^1$.
\end{proof}


Suppose now that~$y \in \ell^1$ and that~$\varepsilon > 0$.
Then there exist a sequence~$y' \in \ell^1$ with finite support such that~$\norm{y - y'}_1 < \varepsilon$.
(The sequence~$y'$ results from~$y$ by cutting this sequence off after sufficiently many terms.)
It follows from the above claim that~$y'$ is contained in the range~$A(\ell^1)$, because~$a_n \neq 0$ for every~$n \in \Natural$ and~$y'$ has only finite support.
This shows that~$A(\ell^1)$ is dense in~$\ell^1$.





\subsection{}

If~$\inf_{n \in \Natural} \abs{a_n} > 0$ then it follows from \cref{sequence in image} that~$A(\ell^1) = \ell^1$.
If on the other hand~$\inf_{n \in \Natural} \abs{a_n} = 0$ then we distinguish between two cases:
\begin{itemize}
  \item
    If~$a_n = 0$ for some~$n \in \Natural$ then~$e^{(n)} \notin A(\ell^1)$ and hence~$A(\ell^1) = \ell^1$.
  \item
    Suppose otherwise that~$a_n \neq 0$ for every~$n \in \Natural$.
    Then there exist a subsequence~$(a_{n(k)})_k$ with~$\abs{a_{n(k)}} \leq 1/k$ for every~$k$.
    Let~$y = (y_n)_n \in \ell^1$ be the sequence with~$y_{n(k)} = 1/k^2$ for every~$k$ and $y_n = 0$ otherwise.
    Then~$y$ is not contained in the range~$A(\ell^1)$:
    
    There would otherwise exist a sequence~$x = (x_n)_n \in \ell^1$ with~$Ax = y$.
    It would then follow for every~$k$ that
    \[
            \frac{1}{k^2}
      =     \abs{ y_{n(k)} }
      =     \abs{ a_{n(k)} } \abs{ x_{n(k)} }
      \leq  \frac{1}{k} \abs{ x_{n(k)} } \,,
    \]
    and hence~$\abs{ x_{n(k)} } \geq 1/k$.
    But then~$x \notin \ell^1$, a contradition.
\end{itemize}





\subsection{}

Suppose that~$A \in \Compact(\ell^1)$.
Then~$C \defined \closure{A(\ball{2}{0})}$ is compact.
But suppose that also~$\abs{a_n} \nto 0$.
Then there exists for some~$\varepsilon > 0$ a subsequence~$(\abs{a_{n(k)}})_k$ with~$\abs{a_{n(k)}} > \varepsilon$ for every~$k$.
Then
\[
    \norm{ A e_{n(k)} - A e_{n(k')} }
  = \abs{ a_{n(k)} } + \abs{ a_{n(k')} }
  > 2 \varepsilon
\]
for all~$k' > k$.
This shows that the sequence~$( A e_{n(k)} )_k$ in~$C$ has not subsequence that is Cauchy, and hence no subsequence that is convergent.
But this contradicts the compactness of~$C$.

Suppose on the other hand that~$a_n \to 0$ let~$\varepsilon > 0$.
To show that~$\closure{A(\ball{1}{0})}$ is compact it sufficies to show that~$A(\ball{1}{0})$ is precompact because~$\ell^1$ is complete.
So let~$\varepsilon > 0$.
We need to show for~$B \defined \ball{0}{1}$ that~$A(B)$ is covered by finitely many~\dash{$\varepsilon$}{balls}.

It follows from~$a_n \to 0$ that there exist some~$N$ with~$\abs{a_n} < \varepsilon/4$ for all~$n > N$.
Let~$C \defined \max(\abs{a_1}, \dotsc, \abs{a_N}, 1)$.
It then holds for all~$x, y \in B$ that
\begin{equation}
  \label{estimate}
  \begin{aligned}
          \norm{Ax - Ay}_1
    &=    \sum_{n=1}^\infty \abs{a_n} \abs{x_n-y_n} \\
    &=      \sum_{n=1}^N \abs{a_n} \abs{x_n - y_n}
          + \sum_{n=N+1}^\infty \abs{a_n} \abs{x_n - y_n} \\
    &\leq   \sum_{n=1}^N C \abs{x_n - y_n}
          + \sum_{n=N+1}^\infty \frac{\varepsilon}{4} \abs{x_n - y_n} \\
    &\leq   C \sum_{n=1}^N \abs{x_n - y_n}
          + \frac{\varepsilon}{4} \norm{x-y}_1  \\
    &\leq   C \sum_{n=1}^N \abs{x_n - y_n}
          + \frac{\varepsilon}{4} (\norm{x}_1 + \norm{y}_1) \\
    &\leq   C \sum_{n=1}^N \abs{x_n - y_n}
          + \frac{\varepsilon}{2} \,.
  \end{aligned}
\end{equation}
The normed vector space~$(\Real^N, \norm{\,\cdot\,}_1)$ is \dash{finite}{dimesional} and complete, so its unit ball~$B' \defined \ball{1}{0}$ us precompact.
Hence there exist finitely many~$x'_1, \dotsc, x'_n \in B'$ such that for every~$y' \in B'$ there exists some index~$i$ with~$\norm{x'_i - y'} < \varepsilon/(2C)$.
By padding the vectors~$x'_1, \dotsc, x'_r$ with zeroes we get sequences~$x_1, \dotsc, x_r \in \ell^1$ with~$\norm{x_i}_1 = \norm{x'_i}_1 < 1$ and hence~$x_i \in B$.

For a sequence~$y \in B$ the truncated vector~$y' = (y_1, \dotsc, y_N) \in \Real$ is contained in the unit ball~$B'$ because~$\norm{y'}_1 \leq \norm{y}_1 \leq 1$.
Hence there exist some index~$i$ with~$\norm{x'_i - y'} < \varepsilon/(2C)$.
The calculation~\eqref{estimate} then shows that
\[
        \norm{A x_i - A y}_1
  \leq  C \norm{x'_i - y'}_1 + \frac{\varepsilon}{2}
  <     C \frac{\varepsilon}{2C} + \frac{\varepsilon}{2}
  =     \varepsilon \,.
\]
Hence we find that~$A(B)$ is covered by the finitely many open balls~$\ball{\varepsilon}{Ax_i}$.




