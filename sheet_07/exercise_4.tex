\section{}





\subsection{}
For every~$x \in X$ let
\[
            I_x
  \defined  \{
              \alpha > 0
            \suchthat
              a^{-1} x \in C
            \} \,.
\]
It follows from~$C$ being convex and containing~$0$ that for every~$\alpha \in I_x$ also~$[\alpha,\infty) \subseteq I_x$.
This shows that~$I_x$ is an interval, namely a subinterval of~$(0,\infty)$ that is unbounded from above.

It follows from~$0$ being contained in the interior of~$C$ that there exist some~$\varepsilon > 0$ with~$\cball{\varepsilon}{0} \subseteq C$.
It then holds for every nonzero~$x \in X$ that~$(\varepsilon/\norm{x}) x \in C$ and hence~$\norm{x}/\varepsilon \in X$.
It follows for~$M \defined 1/\varepsilon$ that
\[
        p(x)
  =     \inf I_x
  \leq  \frac{1}{\varepsilon} \norm{x}
  =     M \norm{x} \,.
\]
For on the other hand~$x = 0$ we have that~$I_x = (0,\infty)$ and hence~$p(0) = 0$.
This shows that~$p$ is~{\welldef} with ~$0 \leq p \leq M \norm{x} < \infty$ for every~$x \in X$.





\subsection{}

It holds for every~$x \in X$ and every nonzero~$\lambda > 0$ that~$I_{\lambda x} = \lambda I_x$.
It follows that
\[
    p(\lambda x)
  = \inf I_{\lambda x}
  = \inf (\lambda I_x)
  = \lambda \inf I_x
  = \lambda p(x)  \,,
\]
where we again use that~$\lambda > 0$.

To show that~$p$ is subaddive let~$x, y \in X$.
It sufficies to show that~$I_x + I_y \subseteq I_{x+y}$ because then
\[
        p(x+y)
  =     \inf I_{x+y}
  \leq  \inf (I_x + I_y)
  =     (\inf I_x) + (\inf I_y)
  =     p(x) + p(y) \,.
\]
So let~$\alpha \in I_x$ and~$\beta \in I_y$.
Then both~$x/\alpha$ and~$y/\beta$ are contained in~$C$ and we want to show that also~$(x+y)/(\alpha + \beta) \in C$.
This holds true because
\[
    \frac{x+y}{\alpha + \beta}
  =   \frac{\alpha}{\alpha + \beta} \cdot \frac{x}{\alpha}
    + \frac{\beta}{\alpha + \beta} \cdot \frac{y}{\beta}
\]
is a convex combination of~$x/\alpha$ and~$y/\beta$ and hence again contained in~$C$.





\subsection{}

For every~$x \in X$ the map~$h_x \colon (0,\infty) \to X$ given by~$h(\alpha) = x/\alpha$ is continuous, hence~$I_x = h^{-1}(C)$ is closed in~$(0,\infty)$.
Since~$I_x$ is also a subinterval of~$(0,\infty)$ we find that either~$I_x = (0,\infty)$ or~$I_x = [p(x), \infty)$.
(Because otherwise~$I_x = (p(x), \infty)$ with~$p(x) > 0$, but this is not a closed subset of~$(0,\infty)$.)
We find in both cases that
\[
        p(x) \leq 1
  \iff  \inf I_x \leq 1
  \iff  1 \in I_x
  \iff  x \in C \,.
\]




