\section{}

We denote as always by~$e^{(k)}$ the sequence with~$e^{(k)}_k = 1$ and~$e^{(k)}_n = 0$ for~$n \neq k$.





\subsection{}



\subsubsection{The case $1 < p < \infty$}

We write~$q \defined p'$.
It follows with Hölder’s inequality that~$J(y)$ is for every sequence~$y = (y_n)_n \in \ell^q$ a {\welldef} linear map~$J(y) \colon \ell^p \to \Real$ with~$\norm{J(y)} \leq \norm{y}_q$.
To show that~$\norm{J(y)} \geq \norm{y}_q$ we may assume that~$\norm{y}_q = 1$ and consider the sequence~$x = (y_n)_n \in \ell^p$ given by~$x_n = \sign(y_n) \abs{y_n}^{q/p}$.
Then~$\norm{x}_p = 1$ and hence
\[
    \abs{J(y)(x)}
  = \sum_{n=1}^\infty \abs{y_n}^{1 + q/p}
  = \sum_{n=1}^\infty \abs{y_n}^q
  = \norm{y}_q^q
  = 1
  = \norm{y}_q \norm{x}_p \,.
\]
We have shown that~$J \colon \ell^q \to (\ell^p)'$ is an isometric linear embedding.

It remains to show that~$J$ is surjective, so let~$\varphi \in (\ell^p)'$.
We note that the linear span~$\gen{e^{(n)} \suchthat n \in \Natural}$ is dense in~$\ell^p$.
The continuous linear map~$\varphi$ is therefore uniquely determined by the values~$y_n \defined \varphi(e^{(n)})$ with~$n \in \Natural$.
To prove that~$\varphi$ is contained in the range~$J(\ell^q)$ we need to show that the squence~$y \defined (y_n)_n$ is contained in~$\ell^q$.
We show that there exist for every~$N \in \Natural$ some~$x_N \in \ell^p$ with~$\norm{x}_p = 1$ such that~$\abs{J(y)(x)} = (\sum_{n=1}^N \abs{y_q}^q)^{1/q}$.
Then
\[
        \norm{y}_q
  =     \lim_{N \to \infty} \left( \sum_{n=1}^N \abs{y_q}^q \right)^{1/q}
  =     \lim_{N \to \infty} \abs{J(y)(x_N)}
  \leq  \lim_{n \to \infty} \norm{J(y)} \norm{x_N}_p
  =     \norm{J(y)}
\]
and hence~$\norm{y}_q < \norm{J(y)} < \infty$.

We fix~$N \in \Natural$.
We note that for~$V \defined (\Real^N, \norm{\,\cdot\,}_q)$ and~$W \defined (\Real^N, \norm{\,\cdot\,}_p)$ we find as above that~$J \colon V \to W'$ is an isometric embedding.
It follows that~$J$ is already an isometric isomorphism because~$\dim V = N = \dim W'$.
It follows for~$y' \defined (y_1, \dotsc, y_N)$ and the compactness of the unit sphere of~$W$ (because~$W$ is {\fd}) that there exist some~$x' \in W$ with~$\norm{x'}_p = 1$ and~$\abs{J(y')(x')} = \norm{y'}_q$.
By padding the vector~$x'$ with zeroes we get a sequence~$x \in \ell^p$ with
\[
    \abs{J(y)(x)}
  = \abs{J(y')(x')}
  = \norm{y'}_q
  = \left( \sum_{n=1}^N \abs{y_n}^q \right)^{1/q} \,,
\]
as desired.



\subsubsection*{The case $p = 1$, $q = \infty$}

We find as before with Hölder’s inequality that~$J \colon \ell^\infty \to (\ell^1)'$ is a {\welldef} linear map with~$\norm{J(y)} \leq \norm{y}_\infty$.
We have on the other hand for~$y = (y_n)_n \in \ell^\infty$ that
\[
        \abs{y_n}
  =     \abs{ J(y)(e^{(n)}) }
  \leq  \norm{J(y)} \norm{ e^{(n)} }_1
  =     \norm{J(y)}
\]
for every~$n \in \Natural$, and hence that~$\norm{y}_\infty \leq \norm{J(y)}$.
This shows that~$J$ is an isometric embedding.

To show that~$J$ is surjective we again pick~$\varphi \in (\ell^1)'$ and need to show that the sequence~$y = (y_n)_n$ with~$y_n \defined \varphi(e^{(n)})_{n \in \Natural}$ for every~$n \in \Natural$ is contained in~$\ell^\infty$.
This holds because
\[
        \abs{y_n}
  =     \abs{ \varphi(e^{(n)}) }
  \leq  \norm{\varphi} \norm{e^{(n)}}_1
  =     \norm{\varphi}
\]
for every~$n \in \Natural$, and hence~$\norm{y}_\infty < \norm{\varphi} < \infty$.





\subsection{}

We assume that~$c_0$ is to be endowed with the norm~$\norm{\,\cdot\,}$, i.e.\ that~$c_0$ is a subspace of~$\ell^\infty$.
We can then proceed as before:

It follows with Hölder’s inequality that~$J \colon \ell^1 \to c_0'$ is a {\welldef} linear map with~$\norm{J(y)} \leq \norm{y}_1$ for every~$y \in \ell^1$.
To see that also~$\norm{J(y)} \geq \norm{y}_1$ we can consider for every~$N \in \Natural$ the sequence~$x = (x_n)_n \in c_0$ with
\[
    x_n
  = \begin{cases}
      \sign(y_n)  & \text{if~$n \leq N$}  \,, \\
      0           & \text{otherwise}      \,.
    \end{cases}
\]
This sequence satisfies~$\norm{x}_\infty \leq 1$ and
\[
        \sum_{n=1}^N \abs{y_n}
  =     \abs{ J(y)(x_N) }
  \leq  \norm{J(y)} \norm{x_N}
  =     \norm{J(y)}
\]
for every~$N \in \Natural$.
Hence~$\norm{y}_1 = \sum_{n=1}^\infty \abs{y_n} \leq \norm{J(y)}$.
This shows that~$J$ is an isometric embedding.

To show that~$J$ is surjective we note that the linear span~$\gen{e^{(n)} \suchthat n \in \Natural}$ is dense in~$c_0$.
(Every sequence~$x \in c_0$ can be approximated by a finite sequence by truncation.)
To show that~$\varphi \in c_0'$ is contained in the range~$J(\ell^1)$ is therefore sufficies (by the same reasoning as before) to show that the sequence~$y = (y_n)_n$ with~$y_n = \varphi(e^{(n)})$ is contained in~$\ell^1$.
This is the case because we have for every~$N \in \Natural$ for the sequences~$x \in c_0$ as above that
\[
        \sum_{n=1}^N \abs{y_n}
  =     \abs*{ \sum_{n=1}^\infty \varphi(e^{(n)}) x_n }
  =     \abs{ \varphi(x) }
  \leq  \norm{\varphi} \norm{x_N}_\infty
  \leq  \norm{\varphi} \,,
\]
and hence~$\norm{y}_1 = \sum_{n=1}^\infty \leq \norm{\varphi} < \infty$.


\begin{remark}
  We can see that the above approch fails to show the (nonexisting) surjectivity of~$\ell^1 \to (\ell^\infty)'$ because the linear span~$\gen{e^{(n)} \suchthat n \geq 0}$ is not dense in~$\ell^\infty$.
\end{remark}




