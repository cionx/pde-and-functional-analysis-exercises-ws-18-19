\section{}





\subsection{}



\subsubsection{Boundedness}

It holds for every~$k$ that
\[
        \norm{f_k}^p
  =     \int_{-1}^1 \abs{\sin(k \pi x)}^p \dd{x}
  \leq  \int_{-1}^1 1 \dd{x}
  =     2 \,.
\]
The sequence~$(f_k)_k$ is therefore bounded in~$\Lp^p([-1,1])$ for~$1 \leq p < \infty$.
The sequence is also bounded in~$\Lp^\infty([-\infty,\infty])$ because~$\norm{f_k} \leq 1$ for every~$k$.



\subsubsection{Weak Convergence}

It follows from part~(i) of Exercise~3 that~$f_k \wto 0$ in~$\Lp^p([-1,1])$ for~$1 < p < \infty$, and that~$f_k \wtostar 0$ in~$\Lp^\infty([-1,1])$, where we use that
\[
    \frac{1}{2} \int_{-1}^1 \sin(k \pi x) \dd{x}
  = 0 \,.
\]

For~$p = 1$ we would like to again use Exercise~3, but this case is not covered there.
But it was claimed in the lecture (Page~69,~(i) at the bottom) that this also works for~$p = 1$.
So I suspect that also~$f_k \wto 0$ in~$\Lp^1([-1,1])$, but I don’t know how to argue for this.



\subsubsection{Strong Convergence}

We have for every~$1 \leq p < \infty$ and~$k \geq 2$ that
\begin{align*}
      \norm{f_{2^{k+2}} - f_{2^{k+1}}}_p^p
  &=  \int_{-1}^1 \abs{ \sin(2^{k+2} \pi x) - \sin(2^{k+1} \pi x) }^p \dd{x}  \\
  &=  2 \int_{-1/2}^{1/2} \abs{ \sin(2^{k+2} \pi x) - \sin(2^{k+1} \pi x) }^p \dd{x}  \\
  &=  \int_{-1}^1 \abs{ \sin(2^{k+1} \pi x) - \sin(2^k \pi x) }^p \dd{x}  \\
  &=  \norm{f_{2^{k+1}} - f_{2^k}}_p^p \,,
\end{align*}
where we use for the second equality that the map~$\abs{\sin(2^{k+2} \pi x) - \sin(2^{k+1} \pi x)}^p$ is~\dash{$(1/2)$}{periodic} (because~$k+1 \geq 3$), and for the third equality the change of variables~$x \to x/2$.
It follows inductively that
\[
    \norm{f_{2^{k+1}} - f_{2^k}}_p
  = \norm{f_8 - f_4}
\]
for all~$k \geq 2$, and hence that the sequene~$(f_k)_k$ is not Cauchy in~$\Lp^p([-1,1])$ because~$\norm{f_8 - f_4}_p > 0$.
This shows that the sequence~$(f_k)_k$ is not strongly convergent in~$\Lp^p([-1,1])$.

We have for every~$k$ that
\[
      f_{2^k}\left( \frac{1}{2^{k+1}} \right)
    - f_{2^{k+1}}\left( \frac{1}{2^{k+1}} \right)
  =   \sin\left( \frac{\pi}{2} \right)
    - \sin(\pi)
  = 1 - 0
  = 1 \,,
\]
and hence
\[
        \norm{ f_{2^{k+1}} - f_{2^k} }_\infty
  \geq  1
\]
because~$f_{2^{k+1}} - f_{2^k}$ is continuous.
This shows that the sequence~$(f_k)_k$ is not Cauchy in~$\Lp^\infty([-1,1])$, and hence not strongly convergent.





\subsection{}


\subsubsection{Boundedness}

We have for every~$1 \leq p < \infty$ that
\begin{align*}
      \norm{f_k}_p^p
  &=  \int_0^1 \abs{f_k(x)}^p \dd{x}
   =  \int_0^{1/k} ( k^{1/4} - k^{5/4} x )^p \dd{x}
   =  \int_0^1 ( k^{1/4} - k^{1/4} x )^p \frac{1}{k} \dd{x} \\
  &=  \int_{-1}^0 (-k^{1/4} x)^p \frac{1}{k} \dd{x}
   =  \int_0^1 (k^{1/4} x)^p \frac{1}{k} \dd{x}
   =  k^{-3/4} \int_0^1 x^p \dd{x}
   =  \frac{k^{-3/4}}{p+1} \,.
\end{align*}
We find that~$\norm{f_k}_p^p \to 0$ as~$k \to \infty$, and hence that the sequence~$(f_k)_k$ is bounded in~$\Lp^p([0,1])$.

The functions~$f_k$ are continuous with~$f_k(0) = k^{1/4}$ and strictly decreasing until they reach~$0$, whence~$\norm{f_k}_\infty = k^{1/4}$.
The sequence~$(f_k)_k$ is therefore not bounded in~$\Lp^\infty([0,1])$.



\subsubsection{Weak Convergence}

It follows for~$1 \leq p < \infty$ from~$f_k \to 0$ (which we show below) that~$f_k \wto 0$.
The function~$(f_k)_k$ is unbounded in~$\Lp^\infty([0,1])$, and is therefore not {\weaklystar} convergent.



\subsubsection{Strong Convergence}

We have seen above that~$\norm{f_k}_p \to 0$ for~$1 \leq p < \infty$, and hence~$f_k \to 0$.
The sequence~$(f_k)_k$ is unbounded in~$\Lp^\infty([0,1])$, and hence not strongly convergent.





\subsection{}

The support of~$\varphi$ is bounded and hence compact, and~$\varphi$ is continuous.
We therefore find that~$\varphi \in \Lp^p(\Real)$ for every~$1 \leq p \leq \infty$.

If~$\varphi = 0$ then~$f_k = 0$ for every~$k$.
Then~$f_k \to 0$ and consequently also~$f_k \wto 0$, and the sequence~$(f_k)_k$ is bounded in~$\Lp^p(\Real)$.
So let’s assume in the following that~$\varphi \neq 0$.
Then~$\norm{\varphi}_p > 0$ for every~$1 \leq p \leq \infty$ by the continuity of~$\varphi$.



\subsubsection{Boundedness}

We have~$\norm{f_k}_p =  \norm{\varphi}_p$ for every~$1 \leq p \leq \infty$, whence the sequence~$(f_k)_k$ is bounded in~$\Lp^p(\Real)$ for very~$1 \leq p \leq \infty$.



\subsubsection{Weak Convergence}

If~$1 < p < \infty$ then there exist for every~$f' \in \Lp^p(\Real)'$ some~$g \in \Lp^q(\Real)$ (where~$p$ and~$q$ are dual exponents) with~$\bil{\tilde{f},f'} = \int_{\Real} \tilde{f} g$ for every~$\tilde{f} \in \Lp^p(\Real)$.
It follows that
\[
        \abs{ \bil{f_k, f'} }
  =     \abs*{ \int_{\Real} f_k g }
  =     \abs*{ \int_{[k-1,k+1]} f_k g }
  \leq  \int_{[k-1,k+1]} \abs{f_k} \abs{g}
  \leq  \norm{\varphi}_\infty \int_{[k-1,k+1]} \abs{g}
  \to   0 \,.
\]
Here we use that~$\Lp^q([k-1,k+1]) \subseteq \Lp^1([k-1,k+1])$ because the compact interval~$[k-1,k+1]$ has finite measure.
This shows that~$f_k \wto 0$ if~$1 < p < \infty$.

For~$p = 1$ we can consider a~\dash{$4$}{periodic} function~$g \in \Lp^\infty(U)$ with
\[
    \restrict{g}{(-1,1)}
  = \restrict{\sign(\varphi)}{(-1,1)}
  \quad\text{and}\quad
    \restrict{g}{(1,3)}
  = 0 \,.
\]
(We do not care about the values~$g(n)$ with~$n \in \Integer$.)
We consider the resulting functional~$f' \in \Lp^1(\Real)'$ given by
\[
  \bil{f,f'}
  \defined
  \int_\Real f g \,.
\]
We have that
\[
    \bil{f_{4k}, f}
  = \int_{\Real} f_{4k} f_{4k} g
  = \int_{[4k-1,4k+1]} f_{4k} g
  = \int_{-1,1} \varphi g
  = \int_{-1,1} \abs{\varphi}
  = \norm{\varphi}_1
  > 0 \,,
\]
and
\[
    \bil{f_{4k+2}, f}
  = \int_{\Real} f_{4k+2} f_{4k} g
  = \int_{[4k+1,4k+3]} f_{4k} g
  = \int_{1,3} f_2 g
  = \int_{-1,1} 0
  = 0 \,.
\]
This shows that the sequence~$(\bil{f_k, f'})_k$ does not converge, whence that the sequence~$(f_k)_k$ does not converge weakly.

Consider now~$p = \infty$.
We have for every~$f \in \Lp^1(\Real)$ that
\begin{align*}
        \abs*{ \int_{\Real} f_k f }
  &\leq \int_{\Real} \abs{f_k} \abs{f}
   =    \int_{[k-1,k+1]} \abs{f_k} \abs{f}  \\
  &\leq \int_{[k-1,k+1]} \norm{\varphi} \abs{f}
   \leq \norm{\varphi}_\infty \int_{[k-1,k+1]} \abs{f}
   \leq \norm{\varphi}_\infty \int_{k-1}^\infty \abs{f}
   \to  0 \,,
\end{align*}
which shows that~$f_k \wtostar 0$ in~$\Lp^\infty(\Real)$.




\subsubsection{Strong Convergence}

We have for all~$k$ that $f_k$ and~$f_{k+3}$ have disjoint support, because~$\supp(f_k) \subseteq \ball{1}{k}$.
Hence
\[
    \norm{f_k - f_{k+3}}_p^p
  = \norm{f_k}^p + \norm{f_{k+3}}_p^p
  = 2 \norm{\varphi}_p^p
  > 0 \,.
\]
for all~$k$ and every~$1 \leq p < \infty$.
This shows that the sequence~$(f_k)_k$ is not Cauchy in~$\Lp^p(\Real)$ for~$1 \leq p < \infty$, and hence not strongly convergent.
We similarly find that~$\norm{f_k - f_{k+3}}_\infty = \norm{\varphi}_\infty$ for all~$k$, and hence that the sequence~$(f_k)_k$ is also not strongly convergent in~$\Lp^\infty(\Real)$.





\subsection{}

If~$\varphi = 0$ then~$f_k = 0$ for every~$k$.
The sequence~$(f_k)_k$ is then bounded, and~$f_k \to 0$ and~$f_k \wto 0$.
We will therefore assume in the following that~$\varphi \neq 0$.



\subsubsection{Boundedness}

We have for every~$1 \leq p < \infty$ that
\begin{align*}
      \norm{f_k}_p^p
   =  \int_{\Real^n} \abs*{ k^{-n/2} \varphi\left( \frac{x}{k} \right) }^p \dd{x}
  &= \int_{\Real^n} \abs{ k^{-n/2} \varphi(x) }^p k^n \dd{x}  \\
  &= k^{n - np/2} \int_{\Real^n} \abs{\varphi(x)}^p \dd{x}
   = k^{n(1 - p/2)} \norm{\varphi}_p^p \,.
\end{align*}
This shows that the sequence~$(f_k)_k$ is bounded in~$\Lp^p(\Real^n)$ if and only if~$1-p/2 \leq 0$, i.e.\ if and only if~$p \geq 2$.

We have that~$\norm{f_k}_\infty = k^{-n/2} \norm{\varphi}_{\infty}$, and~$k^{-n/2} \to 0$ as~$k \to \infty$.
The sequence~$(f_k)_k$ is therefore bounded in~$\Lp^\infty(\Real^n)$.



\subsubsection{Weak Convergence}

If~$p < 2$ then the sequence~$(f_k)_k$ is unbounded because~$\norm{f_k}_p^p = k^{n(1-p/2)} \norm{\varphi}_p^p \to \infty$, and hence not weakly convergent.

If~$p > 2$ then~$f_k \to 0$ (see below) and hence~$f_k \wto 0$ (resp.~$f_k \wtostar 0$).
It similarly follows from~$f_k \to 0$ in~$\Lp^\infty(\Real^n)$ (see again below) that~$f_k \wtostar 0$.

For~$p = 2$ we show that~$f_k \wto 0$.
For this we use that~$\Lp^2(\Real^n)' = \Lp^2(\Real^n)$, and need to show to
\[
      \int_{\Real^n} f_k g
  \to 0
\]
for every~$g \in \Lp^2(\Real^n)$.
By part~(i) of Exercise~2 is sufficies to show this for~$g \in \cont^\infty_c(\Real^n)$ because~$\cont^\infty_c(\Real^n)$ is dense in~$\Lp^2(\Real^n)$.
We have for every~$g \in \cont^\infty_c(\Real^n)$ that
\begin{align*}
        \int_{\Real^n} f_k(x) g(x) \dd{x}
  &=    k^{-n/2} \int_{\Real^n} \varphi\left( \frac{x}{k} \right) g(x) \dd{x} \\
  &=    k^{-n/2} \int_{\Real^n} \varphi(x) g(kx) k^n \dd{x}
   =    \int_{\Real^n} \varphi(x) g_k(x) \dd{x}
\end{align*}
where~$g_k(x) \defined k^{n/2} g(kx)$.
We have seen in the lecture (last bullet point, part (ii), before Theorem~8.4 in the lecture notes) that
\[
      \int_{\Real_n} \varphi(x) g_k(x)
  \to 0 \,,
\]
as desired.





\subsubsection{Strong Convergence}

We have seen above that the sequence~$(f_k)_k$ is for~$p < 2$ unbounded, and hence not strongly convergent.

We also find for~$2 < p < \infty$ that~$\norm{f_k}_p^p = k^{n(1-p/2)} \norm{\varphi}_p^p \to 0$ and hence~$f_k \to 0$.
We have seen above that~$\norm{f_k}_\infty = k^{-n/2} \norm{\varphi}_\infty \to 0$, whence~$f_k \to 0$ in~$\Lp^\infty(\Real)$.

If the sequence of~$(f_k)_k$ would strongly converge in~$\Lp^2(\Real^2)$, say~$f_k \to f$, then also~$f_k \wto f$.
We have seen above that~$f_k \wto 0$, hence we would have that~$f = 0$.
But we have seen that~$\norm{f_k}_2 = \norm{\varphi}_2 > 0$ for every~$k$, so~$f \nto 0$.
This shows that the sequence~$(f_k)_k$ does not strongly converge in~$\Lp^2(\Real^n)$.
















