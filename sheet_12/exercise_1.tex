\section{}





\subsection{}

We show that every operator~$T \in \closure{ \Compact(X,Y) }$ already compact.
We do so by showing that every sequence~$(y_n)_n$ in~$T(\ball{1}{0})$ has a subsequence which converges in~$Y$.
For this it sufficies to show that for every~$\varepsilon > 0$ there exists a subsequence~$(y_{n_k})_k$ with~$\norm{y_{n_k} - y_{n_l}} < \varepsilon$ for all~$k$,~$l$.
It then follows with the diagonal trick that there alsoa exists a subsequence~$(y_{n_k})_k$ with~$\norm{y_{n_k} - y_{n_l}} < 1/k$ for all~$k$.
This subsequence is then a Cauchy sequence and hence convergent by the completeness of~$Y$.

It follows from the assumption~$T \in \closure{ \Compact(X,Y) }$ that there exists a compact operator~$T' \in \Compact(X,Y)$ with~$\norm{T - T'} < \varepsilon / 3$.
Let~$(x_n)_n$ be a sequence in~$\ball{1}{0}$ with~$y_n = T(x_n)$ and let~$(y'_n)_n$ be the sequence in~$T'( \ball{1}{0} )$ given by~$y'_n = T' x_n$ for all~$n$.
It follows from the compactness of the operator~$T'$ that there exists a subsequence~$(y'_{n_k})_k$ that converges in~$Y$, and we may assume that
\[
  \norm{ y'_{n_k} - y'_{n_l} } < \frac{\varepsilon}{3}
\]
for all~$k$,~$l$.
It follows for the corresponding subsequence~$(y_{n_k})_k$ that
\begin{align*}
  \norm{ y_{n_k} - y_{n_l} }
  &\leq
    \norm{ y_{n_k} - y'_{n_k} }
  + \norm{ y'_{n_k} - y'_{n_l} }
  + \norm{ y'_{n_l} - y_{n_l} }
  \\
  &\leq
    \norm{ T x_{n_k} - T' x_{n_k} }
  + \norm{ y'_{n_k} - y'_{n_l} }
  + \norm{ T' x_{n_l} - T x_{n_l} }
  \\
  &\leq
    \norm{T - T'} \norm{x_{n_k}}
  + \norm{ y'_{n_k} - y_{n_l} }
  + \norm{T - T'} \norm{x_{n_l}}
  \\
  &\leq
    \frac{\varepsilon}{3} \cdot 1
  + \frac{\varepsilon}{3}
  + \frac{\varepsilon}{3} \cdot 1
  =
  \varepsilon \,.
\end{align*}





\subsection{}

If~$H$ is finite dimensional then
\[
  \{
    T \in \Lin(H)
  \suchthat
    \dim \Range(T) < \infty
  \}
  =
  \Lin(H)
  =
  \Compact(H) \,.
\]
We will therefore consider in the following only the case that~$H$ is \dash{infinite}{dimensional}.


\begin{lemma}
  \label{uniform on compact}
  Let~$X$ and~$Y$ be normed spaces and let~$(T_n)_n$ be a bounded sequence in~$\Lin(X,Y)$ that converges pointwise to~$T \in \Lin(X,Y)$.
  Then~$T_n \to T$ uniformly on every compact subset of~$X$.
\end{lemma}


\begin{proof}
  Let~$K \subseteq X$ be a compact subset.
  There exists by assumption some constant~$C > 0$ with~$\norm{T_n} < C$ for all~$n$ and also~$\norm{T} < C$.
  There exists for every~$\varepsilon > 0$ finitely many~$x_1, \dotsc, x_k \in X$ with
  \[
    K
    \subseteq
    \ball{\varepsilon}{x_1} \cup \dotsb \cup \ball{\varepsilon}{x_n} \,.
  \]
  There exists some~$N$ with~$\norm{T_n x_i - T x_i} < \varepsilon$ for every~$i \geq N$.
  There hence exists for every~$x \in K$ some~$i$ with~$\norm{x_i - x} < \varepsilon$, and it follows that
  \begin{align*}
    \norm{T_n x - T x}
    &=
      \norm{ T_n x - T_n x_i }
    + \norm{ T_n x_i - T x_i }
    + \norm{ T x_i - T x }
    \\
    &\leq
      \norm{T_n} \norm{x - x_i}
    + \norm{ T_n x_i - T x_i }
    + \norm{T} \norm{x - x_i}a
    \\
    &\leq
    (2 C + 1) \varepsilon
  \end{align*}
  for every~$i \geq N$.
  This shows that~$T_n \to T$ uniformly on~$K$.
\end{proof}

If~$T$ is an operator of finite rank then~$\closure{T(\ball{1}{0})}$ is a subset of the finite dimensional normed space~$\Range(T)$ that is both closed and bounded, and hence compact by Heine--Borel.
This shows that
\[
  \{
    T \in \Lin(H)
  \suchthat
    \dim \Range(T) < \infty
  \}
  \subseteq
  \Compact(H)
\]
and hence that
\[
  \closure{
  \{
    T \in \Lin(H)
  \suchthat
    \dim \Range(T) < \infty
  \}
  }
  \subseteq
  \Compact(H)
\]
because~$\Compact(H)$ is closed a subspace of~$\Lin(H)$.

Suppose now that~$T \in \Compact(H)$ is a compact operator.
Let~$(e_n)_n$ be an orthonormal basis for~$H$, and for every~$n$ let
\[
  P_n
  \colon
  H
  \to
  H \,,
  \quad
  x
  \mapsto
  \sum_{i=1}^n (x,e_i) e_i
\]
be the orthogonal projection onto the finite dimenisonal subspace~$\gen{e_1, \dotsc, e_n}$;
this linear operator is continuous because the linear functionals~$(-,e_i)$ are continuous.
Then~$P_n \to \Id$ pointwise because~$x = \sum_{i=1}^\infty (x, e_i) e_i$ for every~$x \in H$ (this is one of the equivalent characterizations of an orthonormal basis that were given in the lecture).
It follows from \cref{uniform on compact} that~$P_n \to \Id$ uniformly on~$\closure{T(\ball{1}{0})}$.
We find for the linear operators~$T_n \defined P_n \circ T$, which have finite dimensional range, that
\[
  T_n
  =
  P_n \circ T
  \to
  {\Id} \circ T
  = T
\]
uniformly on~$X$, and hence that~$T_n \to T$ in~$\Lin(H)$.
This shows that~$T$ can be approximated by finite rank operators and hence that
\[
  \Compact(H)
  \subseteq
  \closure{
  \{
    T \in \Lin(H)
  \suchthat
    \dim \Range(H) < \infty
  \}
  } \,.
\]


\begin{remark}
  We did not need that~$P_n \to \Id$ pointwise everywhere on~$H$, but only on the compact subset~$\closure{T(\ball{1}{0})}$.
  We see from this that we do not need~$H$ to be separable:
  There exists for every~$n \geq 1$ finitely many~$x_1, \dotsc, x_k \in \closure{T(\ball{1}{0})}$ with
  \[
    \closure{T(\ball{1}{0})}
    \subseteq
    \ball{\varepsilon}{x_1} \cup \dotsb \cup \ball{\varepsilon}{x_k} \,.
  \]
  If~$P_n \in \Lin(H)$ denotes the orthogonal projection onto the finite dimensional linear subspace~$\gen{x_1, \dotsc, x_k}$ then there exists for every~$x \in X$ some~$i$ with~$\norm{x - x_i} < \varepsilon$ and hence
  \[
    \norm{P x - x}
    \leq
    \norm{x_i - x}
    <
    \varepsilon \,.
  \]
  Then~$P_n \to \Id$ pointwise on~$\closure{T(\ball{1}{0})}$ and hence again~$P_n \circ T \to T$ uniformly on~$X$.
\end{remark}



