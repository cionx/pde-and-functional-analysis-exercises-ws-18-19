\section{}

For~$H = 0$ we have~$\sigma(p(A)) = \Complex = p(\sigma(A))$.
We hence consider in the following only the case~$H \neq 0$.

\begin{lemma}
  Let~$f_1, \dotsc, f_n \colon X \to X$ be pairwise commuting maps on a set~$X$.
  Then the composition~$f_1 \circ \dotsb \circ f_n$ is invertible if and only if all~$f_i$ are invertible.
\end{lemma}


\begin{proof}
  If every map~$f_i$ is invertible then~$f_1 \circ \dotsb \circ f_n$ is invertible.
  If on the other hand~$f_1 \circ \dotsb \circ f_n$ is invertible then~$f_1$ is surjective and~$f_n$ is injective.
  We may rearrange the terms~$f_1, \dotsc, f_n$ in the composition~$f_1 \circ \dotsb \circ f_n$ however we want.
  With this we then find that every~$f_i$ is injective and also 
that every~$f_i$ is surjective.
\end{proof}



If~$p = a_0 \in \Complex$ then
\begin{gather*}
  \sigma(p(A))
  =
  \sigma(a_0 \Id)
  =
  \{ a_0 \}
\shortintertext{and}
  p(\sigma(A))
  =
  \begin{cases}
    \{ a_0 \} & \text{if~$\sigma(A)$ is nonempty} \,, \\
    \emptyset & \text{otherwise}  \,.
  \end{cases}
\end{gather*}
It apparently holds that the spectrum of a bounded operator is a complex Hilbert space is nonempty, so that
\[
  \sigma(p(A))
  =
  \{ a_0 \}
  =
  p(\sigma(A))
\]
as desired.
But I don’t think that we have shown this in the lecture.

We consider now case~$\deg p \geq 1$.
We may assume that~$a_n \neq 0$ so that~$\deg p = n$.
For every~$\lambda \in \Complex$ the polynomial~$p(z) - \lambda$ is again of degree~$n$ with leading coefficient~$a_n$, and may therefore be factorized as
\[
  p(z) - \lambda
  =
  a_n (z - \mu_1) \dotsm (z - \mu_n) \,.
\]
We find with the above lemma that
\begin{align*}
  \lambda \in \sigma(p(A))
  &\iff
  \text{$p(A) - \lambda$ is \dash{non}{invertible}}
  \\
  &\iff
  \text{$a_n (A - \mu_1) \dotsm (A - \mu_n)$ is \dash{non}{invertible}}
  \\
  &\iff
  \text{$A - \mu_i$ is \dash{non}{invertible} for some~$i$}
  \\
  &\iff
  \text{$\mu_i \in \sigma(A)$ for some~$i$}
\end{align*}
We also note that~$\mu_1, \dotsc, \mu_n$ are precisely the solutions to the polynomial equations~$p(z) - \lambda = 0$, and hence the solutions to the equation~$p(z) = \lambda$.
We therefore find that
\begin{align*}
  {}&
  \lambda \in \sigma(p(A))
  \\
  \iff{}&
  \text{there exists~$\mu$ with~$p(\mu) = \lambda$ such that~$\mu \in \sigma(A)$}
  \\
  \iff{}&
  \text{there exists~$\mu \in \sigma(A)$ with~$\lambda = p(\mu)$}
  \\
  \iff{}&
  \lambda \in p(\sigma(A))  \,.
\end{align*}



