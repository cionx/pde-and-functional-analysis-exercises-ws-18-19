\section{}

Suppose first that~$T \in \Compact(X,Y)$, and let~$(x_n)_n$ be a sequence in~$X$ with~$x_n \wto x$ for some~$x \in X$.
We start by showing that the sequence~$(x_n)_n$ has a subsequence~$(x_{n_k})_k$ with~$T x_{n_k} \to T x$.

The sequence~$(x_n)_n$ is bounded because it is weakly convergent, whence contained in a ball~$\ball{r}{0}$ for some~$r > 0$.
It follows from the compactness of~$T$ that~$\closure{ T( \ball{r}{0} ) }$ is compact.
There hence exists a subsequence~$(x_{n_k})_k$ such that~$( T x_{n_k} )_k \to y$ for some~$y \in Y$.
It follows from~$x_n \wto x$ that also~$x_{n_k} \wto x$, and hence that~$T x_{n_k} \wto T x$ because
\[
  y' T x_{n_k}
  =
  (T' y') x_{n_k}
  \to
  (T' y') x
  =
  y' T x
\]
for every~$y' \in Y'$.
We find that the sequence~$(T x_{n_k})_k$ converges strongly to~$y$ and weakly to~$T x$.
It follows that~$Tx = y$ because strong convergence implies weak convergence, and weak limits are unique.
This shows altogether that~$T x_{n_k} \to T x$.

Every subsequence of~$(x_n)_n$ is again weakly convergent to~$x$.
The above argument hence shows that every subsequence of~$(T x_n)_n$ has a subsequence that converges strongly to~$T x$.
This then means that the sequence~$(T x_n)_n$ already converges strongly to~$T x$.

Suppose now that~$T x_n \to Tx$ for every sequence~$(x_n)_n \subseteq X$ and every~$x \in X$ with~$x_n \wto x$.
We need to show that~$T$ is continuous and that~$\closure{ T( \ball{1}{0} ) }$ is compact.
We have for every sequence~$(x_n)_n \subseteq X$ and every~$x \in X$ that
\[
  x_n \to x
  \implies
  x_n \wto x
  \implies
  T x_n \to T x \,,
\]
which shows that~$T$ is continuous.
To show that~$T$ is compact we need to show that every sequence~$(y_n)_n$ in~$T( \ball{1}{0} )$ has a subsequence that converges in~$Y$ (see equivalence~(9.8) in the lecture notes, at the beginning of page 87).
There exists a sequence~$(x_n)_n$ in~$X$ with~$(x_n)_n \subseteq \ball{1}{0}$ such that~$y_n = T x_n$ for every~$n$.
The closed unit ball~$\closure{\ball{1}{0}}$ is weakly compact because~$X$ is reflexive, hence there exists a subsequence~$(x_{n_k})_k$ and some~$x \in X$ with~$x_{n_k} \wto x$.
It follows by assumption that
\[
  y_{n_k}
  =
  T x_{n_k}
  \to
  T x \,,
\]
which shows that the subsequence~$(y_{n_k})_k$ converges in~$Y$.



