\section{}





\subsection{}
\label{needed inequality}

We may assume that~$c \geq 0$, because otherwise we can replace~$c$ by~$-c$ and~$x$ by~$-x$;
we may also assume that~$c \neq 0$ and hence~$c > 0$.
Suppose that not~$\norm{x - c \one}_\infty \geq c$.
Then
\[
    \sup_{n \in \Natural} \abs{x_n - c}
  = \norm{x - c \one}_\infty
  < c
\]
and hence there exist~$0 < c' < c$ with~$\abs{x_n - c} < c'$ for every~$n \in \Natural$.
It follows from~$x \in W$ that there exists some~$y \in \ell^\infty$ with~$x = S(y) - y$, and hence~$x_n = y_{n+1} - y_n$ for every~$n \in \Natural$. 
We now have for every~$n \in \Natural$ that
\begin{gather*}
  \abs{y_{n+1} - y_n - c} < c'
\shortintertext{and hence}
  y_{n+1} - y_n \in (c - c', c + c') \,.
\end{gather*}
It follows from~$c' < c$ that~$y_{n+1} - y_n > c-c' > 0$ for every~$n \in \Natural$.
The sequence~$y$ is therefore strictly increasing.
But this contradicts~$y$ being bounded.





\subsection{}

It follows from part~\ref{needed inequality} that~$\one \notin W$ because for~$x = \one$ and~$c = -1$ we get
\[
    \norm{x - c \one}_\infty
  = \norm{0}_\infty
  = 0
  < 1
  = \abs{c} \,.
\]
The sum~$Y \defined W + \Real \one$ is therefore direct, i.e.\ that~$Y = W \oplus \Real \one$.
We can hence define a map
\[
          \varphi
  \colon  Y
  \to     \Real \,,
  \quad   x + c \one
  \mapsto c
\]
where~$x \in W$ and~$c \in \Real$.
The map~$\varphi$ is linear, and it follows from part~\ref{needed inequality} that
\[
        \varphi(y)
  =     \varphi(x + c \one)
  =     c
  \leq  \abs{c}
  \leq  \norm{x + c \one}_\infty
  =     \norm{y}_\infty
\]
for every~$y \in Y$ with~$y = x + c \one$, where~$x \in W$ and~$c \in \Real$.
Hence~$\varphi \in Y'$ with~$\norm{\varphi} \leq 1$.
By considering~$y = \one$ we see that actually~$\norm{\varphi} = 1$ because~$\varphi(\one) = 1$.

It follows from the Hahn--Banch theorem (version~II, Theorem~6.4 from the lecture) that there exist an extension~$\Phi \in (\ell^\infty)'$ of~$\varphi$ with~$\norm{\Phi} = \norm{\varphi} = 1$.
It again holds that~$\Phi(\one) = \varphi(\one) = 1$.
The functional~$\Phi$ is shift invariant:
We have for every~$x \in \ell^\infty$ that
\[
        \Phi(S(x)) = \Phi(x)
  \iff  \Phi(S(x)) - \Phi(x) = 0
  \iff  \Phi(S(x) - x) = 0 \,,
\]
and the last condition holds because~$S(x) - x \in W$ and hence
\[
    \Phi(S(x) - x)
  = \varphi(S(x) - x)
  = 0
\]
by construction of~$\Phi$ via~$\varphi$ and the definition of~$\varphi$.

That~$\Phi$ is positive, i.e.\ that~$\Phi(x) \geq 0$ for~$x \geq 0$, follows from the upcoming part~\ref{estimate on behavior}, because~$\liminf_{n \to \infty} x_n \geq 0$.





\subsection{}
\label{estimate on behavior}


Suppose first that~$x \geq 0$.
We then have for every~$N \in \Integer$ that
\[
        \Phi(x)
  =     \Phi(S^N x)
  \leq  \abs{ \Phi(S^N x) }
  \leq  \norm{ S^N x }_{\infty}
  =     \sup_{n \in \Natural} \abs{x_{n+N}}
  =     \sup_{n \geq N} \abs{x_n}
  =     \sup_{n \geq N} x_n \,,
\]
and hence
\[
        \Phi(x)
  \leq  \limsup_{n \to \infty} x_n \,.
\]
If now~$x \in \ell^\infty$ then we get for~$c \defined \inf_{n \in \Natural} x_n$ that~$x - c \one \geq 0$ and hence
\[
        \Phi(x) - c
  =     \Phi(x - c \one)
  \leq  \limsup_{n \to \infty} (x_n - c)
  =     \left( \limsup_{n \to \infty} x_n \right) - c \,,
\]
which shows that
\[
        \Phi(x)
  \leq  \limsup_{n \to \infty} x_n \,.
\]
We now also find that
\[
        -\Phi(x)
  =     \Phi(-x)
  \leq  \limsup_{n \to \infty} (-x_n)
  =     - \liminf_{n \to \infty} x_n \,,
\]
and hence
\[
        \Phi(x)
  \geq  \liminf_{n \to \infty} x_n \,.
\]










