\section{}





\subsection{}

If~$(X,d)$ is complete then every Cauchy~sequence~$(x_n)_n \subseteq X$ converges in~$X$;
this holds then in particular for every Cauchy~sequence~$(x_n)_n \subseteq A$.

Suppose that every Cauchy sequence~$(x_n)_n \subseteq A$ converges in~$X$, and let~$(y_n)_n \subseteq X$ be a Cauchy sequence.
To show that~$(y_n)_n$ converges in~$X$ we may replace~$(y_n)_n$ by any of its subsequences and therefore assume that~$d(y_l, y_k) \geq 1/k$ for all~$l \geq k$.
There exists for every~$n$ some~$x_n \in A$ with~$d(x_n, y_n) < 1/n$.
The sequence~$(x_n)_n$ is again a Cauchy sequence because it holds for all~$l \geq k$ that
\[
    d(x_l, x_k)
  = d(x_l, y_l) + d(y_l, y_k) + d(y_k, x_k)
  < \frac{1}{l} + \frac{1}{l} + \frac{1}{k}
  < \frac{3}{k} \,.
\]
It follows by assumption that the sequence~$(x_n)_n$ converges in~$X$ to some~$y \in X$.
It follows that
\[
          d(y_n, y)
  =       d(y_n, x_n) + d(x_n, y)
  \leq    \frac{1}{n} + d(x_n, y)
  \longto 0
\]
as~$n \to \infty$, which shows that~$(y_n)_n$ converges in~$X$.





\subsection{}
\label{extension of uniformly continuous maps}

\begin{lemma}
  \label{uniformly continuous preserves cauchy}
  Let~$A$ and~$Y$ be metric spaces and let~$f \colon A \to Y$ be a uniformly continuous map.
  Let~$(a_n)_n \subseteq X$ be a sequence.
  \begin{enumerate}
    \item
      If~$(a_n)_n$ is a Cauchy~sequence then the sequence~$(f(a_n))_n \subseteq Y$ is again a Cauchy~sequence.
    \item
      Suppose that~$f(a_n) \to y$ for some~$y \in Y$.
      Let~$(a'_n)_n \subseteq A$ be another sequence, for which~$d(a'_n, a_n) \to 0$.
      Then also~$f(a'_n) \to y$.
  \end{enumerate}
\end{lemma}

\begin{proof}
  \leavevmode
  \begin{enumerate}
    \item
      For every~$\varepsilon > 0$ there exists some~$\delta > 0$ with~$d(f(a),f(a')) < \varepsilon$ for all~$a, a' \in X$ with~$d(a, a') < \delta$.
      There exists some~$k$ with~$d(a_l, a_{l'}) < \delta$ for all~$l, l' \geq k$, and hence~$d(f(a_l), f(a_{l'})) < \varepsilon$ for all~$l, l' \geq k$.
    \item
      There exists for every~$\varepsilon > 0$ some~$\delta > 0$ with~$d(f(a),f(a')) < \varepsilon/2$ for all~$a, a' \in A$ with~$d(a,a') < \delta$.
      There exists some $N$ with~$d(a_n, a'_n) < \delta$ and~$d(f(a_n), y) < \varepsilon/2$ for all~$n \geq N$.
      It follows that
      \[
          d(f(a'_n), y)
        = d(f(a'_n), f(a_n)) + d(f(a_n), y)
        < \frac{\varepsilon}{2} + \frac{\varepsilon}{2}
        = \varepsilon
      \]
      for all~$n \geq N$, and hence~$f(a'_n) \to y$.
    \qedhere
  \end{enumerate}
\end{proof}

For~$x \in X$ there exist a sequence~$(a_n)_n \subseteq A$ with~$a_n \to x$.
It follows from \cref{uniformly continuous preserves cauchy} that the sequence~$(f(a_n)_n)_n$ is again a Cauchy sequence, hence converges by the completeness of~$Y$.
Let~$\tilde{f}(x) \defined \lim_{n \to \infty} f(a_n)$.

If~$(a'_n)_n \subseteq A$ is another sequence with~$x'_n \to x$ then
\[
        d(a_n, a'_n)
  \leq  d(a_n, x) + d(x, a'_n)
  \to   0
\]
as~$n \to \infty$ and therefore~$f(a'_n) \to \tilde{f}(x)$ by \cref{uniformly continuous preserves cauchy}.
This shows that~$\tilde{f}(x)$ is independent of the choice of sequence~$(a_n)_n$.
(We wont’t actually need this independence, but the author thinks that it is nevertheless important enough to warrant a proof.)

To show that~$\tilde{f}$ is continuous we show that~$\tilde{f}$ is uniformly continuous:
So let~$\varepsilon > 0$ and let~$\delta > 0$ with
\[
    d(f(a), f(a'))
  < \frac{\varepsilon}{3}
\]
for all~$a, a' \in A$ with~$d(a,a') < 3\delta$.
For~$x, x' \in X$ with~$d(x,x') < \delta$ we show that~$\tilde{f}(x,x') < \varepsilon$:
We choose sequences~$(a_n)_n, (a'_n)_n \subseteq A$ with~$a_n \to x$ and~$a'_n \to x'$.
It then holds that~$f(a_n) \to x$ and~$f(a'_n) \to x'$ as seen above.
Let~$n$ be large enough so that
\[
    d(a_n, x),
    d(a'_n, x)
  < \delta
  \quad\text{and}\quad
    d(f(a_n), \tilde{f}(x)),
    d(f(a'_n), \tilde{f}(x))
  < \frac{\varepsilon}{3} \,.
\]
It then follows from the inequalities $d(a_n, x), d(x, x'), d(x', a'_n) < \delta$ that~$d(a_n, a'_n) < 3 \delta$, and hence
\[
  d(f(a_n), f'(a_n)) < \frac{\varepsilon}{3}
\]
by choice of~$\delta$.
We altogether find that
\begin{align*}
          d(\tilde{f}(x), \tilde{f}(x'))
  &\leq   d(\tilde{f}(x), f(a_n))
          + d(f(a_n), f(a'_n))
          + d(f(a'_n), \tilde{f}(x')  \\
  &<      \frac{\varepsilon}{3}
        + \frac{\varepsilon}{3}
        + \frac{\varepsilon}{3}
  =     \varepsilon \,.
\end{align*}
This shows that~$\tilde{f}$ is indeed uniformly continuous.




We also note that the continuous extension~$\tilde{f}$ is unique because~$A$ is dense in~$X$.





\subsection{}

The isometry~$f$ extends by part~\ref{extension of uniformly continuous maps} (uniquely) to a continuous map~$\tilde{f} \colon X \to Y$.
For all~$x, x' \in X$ there exist sequences~$(a_n)_n, (a'_n)_n \subseteq A$ with~$a_n \to x$ and~$a'_n \to x'$.
It then follows from the continuity of~$\tilde{f}$ and the continuity of the metrics~$d_X \colon X \times X \to \Real$ and~$d_Y \colon Y \times Y \to \Real$ (with respect to the product topologies on both~$X \times X$ and~$Y \times Y$) that
\begin{align*}
      d_Y(\tilde{f}(x), \tilde{f}(x'))
  &=  d_Y\left( \lim_{n \to \infty} (\tilde{f}(a_n), \tilde{f}(a'_n)) \right) \\
  &=  \lim_{n \to \infty} d_Y(\tilde{f}(a_n), \tilde{f}(a'_n))  \\
  &=  \lim_{n \to \infty} d_Y(f(a_n), f(a'_n))  \\
  &=  \lim_{n \to \infty} d_A(a_n, a'_n)  \\
  &=  \lim_{n \to \infty} d_X(a_n, a'_n)  \\
  &=  d_X\left( \lim_{n \to \infty} (a_n, a'_n) \right) \\
  &=  d_X(x, x') \,.
\end{align*}
This shows that~$\tilde{f}$ is again an isometry.

It follows that when we restrict~$\tilde{f}$ to a map into its image, then~$\tilde{f}$ become a bijective isometry.%
\footnote{We haven’t explicitely shown that~$\tilde{f}$ is injective, but isometries are always injective.}
This shows that the subspace~$\tilde{f}(X) \subseteq Y$ is complete, and thus closed.
But the image~$\tilde{f}(X)$ is dense in~$Y$ because it contains~$\tilde{f}(A) = f(A)$, which is dense in~$Y$.
It follows that already~$\tilde{f}(X) = Y$.
This shows altogether~$\tilde{f}$ is a bijective isometry.




