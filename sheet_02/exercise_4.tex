\section{}



\subsection{}
\label{criterion for nonseperable}

Suppose there exists a countable dense subset~$D \subseteq X$.
Then there exists for every~$a \in A$ some~$x_a \in D$ with~$d(x_a, a) < \delta/2$.
It follows from~$D$ being countable but~$A$ being uncountable that there exist~$a_1, a_2 \in A$ with~$a_1 \neq a_2$ but~$x_{a_1} = x_{a_2} \defines x$.
It follows that
\[
        d(a_1, a_2)
  \leq  d(a_1, x) + d(x, a_2)
  <     \frac{\delta}{2} + \frac{\delta}{2}
  =     \delta \,,
\]
which contradicts the choice of~$A$.





\subsection{}

For every subset~$S \subseteq \Integer$ there exists a continuous bounded function~$f_S \colon \Real \to \Real$ with
\[
    f_S(n)
  = \begin{cases}
      1 & \text{if~$n \in S$} \,, \\
      0 & \text{otherwise} \,.
    \end{cases}
\]
(One can take for every~$n \in S$ a small hat of height~$1$ with peak at the point~$n$ and support in~$[n - 1/2, n + 1/2]$, and then connects these hats by the zero function.)
It then holds for any two distinct subsets~$S, T \subseteq \Integer$ that some~$n \in \Integer$ is contained in precisely one of the sets~$S$ and~$T$;
it then follows that~$\abs{f_S(n) - f_T(n)} = 1$.
This shows that~$\norm{f_S - f_T} \geq 1$ for all~$S, T \subseteq \mathcal{P}(\Integer)$ with~$S \neq T$ (where~$\mathcal{P}(\Integer)$ denotes the power set of~$\Integer$).
It follows froem the uncountability of the power set~$\mathcal{P}(\Integer)$ and part~\ref{criterion for nonseperable} of the exercise that~$\cont^0_b(\Real)$ is not seperable.





\subsection{}

If~$f_1, f_2 \colon \Real \to \Real$ are functions for which the limits
~$\lim_{x \to \infty} f_1(x)$ and~$\lim_{x \to \infty} f_2(x)$ exist, then it holds for every~$\alpha \in \Real$ that the limit~$\lim_{x \to \infty} (\alpha f_1 + f_2)(x)$ also exists, and is given by
\[
    \lim_{x \to \infty} (\alpha f_1 + f_2)(x)
  =   \alpha \left( \lim_{x \to \infty} f_1(x) \right)
    + \left( \lim_{x \to \infty} f_2(x) \right) \,.
\]
The analogous statement for~$x \to -\infty$ also holds.
It follows that for all functions~$f_1, f_2 \colon \Real \to \Real$ with~$\lim_{\abs{x} \to \infty} f(x) = 0$ and all~$\alpha \in \Real$, the function~$\alpha f_1 + f_2$ again satisfies~$\lim_{\abs{x} \to \infty} (\alpha f_1 + f_2)(x) = 0$.
It also holds that~$0 \in \cont^0_0(\Real)$.
This shows altogether that~$\cont^0_0(\Real)$ is a linear subspace of~$\cont^0_b(\Real)$.

Let~$f \in \cont^0_b(\Real)$ be in the closure of~$\cont^0_0(\Real)$.
Then there exist for every~$\varepsilon > 0$ some~$g \in \cont^0_0(\Real)$ with~$\norm{f-g} < \varepsilon$, and hence
\[
        \abs{f(x)}
  \leq  \abs{g(x)} + \abs{ f(x) - g(x) }
  \leq  \abs{ g(x) } + \varepsilon
\]
for every~$x \in \Real$.
It follows that
\[
        \limsup_{x \to \infty} {\abs{f(x)}}
  \leq  \varepsilon \,.
\]
This shows that~$\limsup_{x \to \infty} \abs{ f(x) } = 0$, which in turn shows that~$\lim_{x \to \infty} f(x) = 0$.
It can be shown in the same way that~$\lim_{x \to -\infty} f(x) = 0$.
This shows together that~$\lim_{\abs{x} \to \infty} f(x) = 0$, and hence that~$\cont^0_0(\Real)$ is closed in~$\cont^0_b(\Real)$.

It remains to show that~$\cont^0_0(\Real)$ is separable.
We know from the lecture that~$\cont^0([-n,n])$ is separable for every~$n \in \Natural$ (see Theorem~2.5 in the lecture notes).
Let~$B_n \subseteq \cont^0([-n,n])$ be a countable dense subset.
We extend every~$g \in B_n$ to a function~$\hat{g} \in \cont^0_0(\Real)$ by letting~$\hat{g}$ tend linearly to~$0$ on the intervals~$[-n-1,-n]$ and~$[n,n+1]$ and setting~$\hat{g} \equiv 0$ outside of~$[-n-1,n+1]$.
This means explicitely that
\[
    \hat{g}(x)
  = \begin{cases}
      0                   & \text{if~$x \leq -n-1$} \,, \\
      g(-n) \cdot (x+n+1) & \text{if~$-n-1 \leq x \leq -n$} \,, \\
      g(x)                & \text{if~$-n \leq x \leq n$}  \,, \\
      g(n) \cdot  (n+1-x) & \text{if~$n \leq x \leq n+1$} \,, \\
      0                   & \text{if~$x \geq n+1$} \,.
    \end{cases} \,.
\]
Let~$\hat{B}_n \defined \{ \hat{g} \suchthat g \in B_n \}$ and set~$\hat{B} \defined \bigcup_{n \geq 0} \hat{B}_n$.
Then~$\hat{B}$ is a countable subset of~$\cont^0_0(\Real)$, and we claim that it is dense:

Let~$f \in \cont_0^0(\Real)$ and let~$\varepsilon > 0$.
There exist some~$n \in \Natural$ with~$\abs{f(x)} \leq \varepsilon$ whenever~$\abs{x} \geq n$.
It holds in particular that
\begin{equation}
  \label{estimate on original function}
        {\abs{f(n)}}, {\abs{f(-n)}}
  \leq  \varepsilon
\end{equation}
There exist by choice of~$B_n$ some~$g \in B_n$ with~$\norm{f-g}_{\cont^0([-n,n])} \leq \varepsilon$.
It follows in particular that~$\abs{f(x) - g(x)} \leq \varepsilon$ for~$x = n, -n$, and hence
\[
        \abs{g(n)}, \abs{g(-n)}
  \leq  2 \varepsilon
\]
by~\eqref{estimate on original function}.
It then follows from the construction of the extension~$\hat{g}$ that~$\abs{g(x)} \leq 2\varepsilon$ for all~$\abs{x} \geq n$.
It follow from the triangle inequality that
\[
        \abs{f(x) - g(x)}
  \leq  \abs{f(x)} + \abs{g(x)}
  \leq  2 \varepsilon + \varepsilon
  =     3 \varepsilon
\]
for all~$\abs{x} \geq n$.
Together with~$\norm{f-g}_{\cont^0([-n,n])} \leq \varepsilon$ this shows that~$\norm{f-g}_{\cont^0_0(\Real)} \leq 3 \varepsilon$.




