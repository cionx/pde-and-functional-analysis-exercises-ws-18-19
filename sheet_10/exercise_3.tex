\section{}

If~$x_n \to x$ then also~$x_n \wto x$, and it follows from the continuity of the norm~$\norm{\,\cdot\,} \colon X \to \Real$ that also~$\norm{x_n} \to \norm{x}$.

Suppose now that~$x_n \wto x$ and~$\norm{x_n} \to \norm{x}$.
If~$x = 0$ then~$\norm{x_n} \to 0$ and hence~$x_n \to 0 = x$.
In the following we will consider the case~$x \neq 0$.

We can replace~$x$ by~$x/\norm{x}$ and~$x_k$ by~$x_k/\norm{x}$ to assume that~$\norm{x} = 1$.
It follows from~$\norm{x_k} \to \norm{x} = 1$ that~$x_k \neq 0$ for all but finitely many~$k$, so we may assume that~$x_k \neq 0$ for every~$k$.
It also follows from~$\norm{x_k} \to \norm{x} = 1$ that
\[
  \norm*{ x_k - \frac{x_k}{\norm{x}} }
  \to
  0 \,;
\]
and we also have for every~$T \in X'$ that
\[
  T\left( \frac{x_k}{\norm{x_k}} \right)
  =
  \frac{T x_k}{\norm{x_k}}
  \to
  \frac{T x}{\norm{x}}
  =
  T x
\]
because~$x_k \wto x$, which shows that also~$x_k/\norm{x_k} \wto x$.
Together this shows that we may also replace~$x_k$ by~$x_k/\norm{x_k}$ to additionally asume that~$\norm{x_k} = 1$ for all~$k$.

We now have that~$\norm{x} = 1$ and~$\norm{x_k} = 1$ for all~$k$, and~$x_k \wto x$.
We have that
\[
  \norm*{ \frac{x_k + x}{2} }
  \leq
  1
\]
for all~$k$, and that~$x_k \to x$ is now equivalent to
\[
  \norm*{ \frac{x_k + x}{2} }
  \to
  1
\]
because~$X$ is uniformly convex.
We know from the lecture (Corollary~6.6) that there exists some~$T \in X'$ with~$\norm{T} = 1$ and~$Tx = \norm{x}$.
It follows that
\[
  1
  \geq
  \norm*{ \frac{x_k + x}{2} }
  \geq
  \abs*{ T\left( \frac{x_k + x}{2} \right) }
  =
  \abs*{ \frac{T x_k + T x}{2} }
  \to
  \abs{ Tx }
  =
  \norm{x} \,
\]
and therefore
\[
  \norm*{ \frac{x_k + x}{2} }
  \to
  1 \,.
\]




