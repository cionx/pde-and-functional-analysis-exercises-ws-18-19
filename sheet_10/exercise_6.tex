\section{}

We assume that the interval~$I$ is nonempty, because otherwise~$F = 0$ independent of the choice of~$f$.
For the length~$\abs{I}$ we hence have~$\abs{I} > 0$.

The map~$F$ is~{\welldef}:
A function~$u \in \Lp^\infty(I)$ is essentially bounded, and thus there exists a compact interval~$J \subseteq \Real$ with~$u(t) \subseteq J$ for almost all~$t \in I$.
The continous map~$f$ is bounded on~$J$, and hence~$f \circ u$ is almost bounded on~$I$.
The composition~$f \circ u$ is measurable because both~$f$ and~$c$ are measurable, and we find that
\[
  \int_I \abs{f \circ u}
  \leq
  \norm{f \circ u}_{\infty} \abs{I}
  <
  \infty  \,.
\]





\subsection{}
\label{continuous iff affine}

Suppose that~$f$ is affine.
Then there exist~$a, b \in \Real$ with~$f(x) = ax + b$ for all~$x \in \Real$.
Suppose that~$(u_k)_k$ is a sequence in~$\Lp^\infty(I)$ and that~$u \in \Lp^\infty(I)$ with~$u_k \wtostar u$.
Then
\[
  \int_I u_k
  =
  \int_I 1 \cdot u_k
  \to
  \int_I 1 \cdot u
  \to
  \int_I u
\]
because~$1 \in \Lp^1(I)$ (since~$I$ is bounded), and therefore
\[
  F(u_k)
  =
  a \int_I u_k + b \abs{I}
  \to
  a \int_I u + b \abs{I}
  =
  F(u) \,.
\]

Suppose now on the other hand that~$F$ is continuous with respect to {\weakstar} convergence.
Let~$x, y \in \Real$ and let~$0 \leq \lambda \leq 1$.
We define for every~$k \geq 1$ a function~$u_k \in \Lp^\infty(I)$ by subdividing the interval~$I$ into~$2k$ subintervals of alternating lengths~$\lambda \abs{I}/(2k)$ and~$(1-\lambda) \abs{I}/(2k)$, and letting~$u_k$ assume the alternating values
\[
  x, y, x, y, \dotsc, x, y
\]
on these subintervals.
We know from Exercise~3, part~(ii) of sheet 9 that
\[
  u_k
  \wto
  \lambda x + (1 - \lambda) y
  \defines
  u \,.
\]
We have that~$\int_I f \circ u_k = \lambda \abs{I} f(x) + (1 - \lambda) \abs{I} f(y)$ for every~$k$, and hence find that 
\begin{align*}
  f( \lambda x + (1-\lambda) y )
  &=
  \frac{1}{\abs{I}} \int_I f \circ u
  =
  F(u)
  =
  \frac{1}{\abs{I}} \lim_{k \to \infty} F(u_k)
  =
  \frac{1}{\abs{I}} \lim_{k \to \infty} \int_I f \circ u_k
  \\
  &=
  \frac{1}{\abs{I}}
  \lim_{k \to \infty}
  \left(
  \lambda \abs{I} f(x) + (1 - \lambda) \abs{I} f(y)
  \right)
  =
  \lambda f(x) + (1 - \lambda) f(y) \,.
\end{align*}
This shows that~$f$ is affine.





\subsection{}

Suppose that~$F$ is lower-semi-continuous with respect to {\weakstar} convergence.
To show that~$f$ is convex we can proceed as in part~\ref{continuous iff affine}, and find that
\begin{align*}
  f( \lambda x + (1-\lambda) y )
  &=
  \frac{1}{\abs{I}} \int_I f \circ u
  =
  F(u)
  \leq
  \frac{1}{\abs{I}} \liminf_{k \to \infty} F(u_k)
  =
  \frac{1}{\abs{I}} \liminf_{k \to \infty} \int_I f \circ u_k
  \\
  &=
  \frac{1}{\abs{I}}
  \liminf_{k \to \infty} \Bigl( \lambda \abs{I} f(x) + (1 - \lambda) \abs{I} f(y) \Bigr)
  =
  \lambda f(x) + (1 - \lambda) f(y) \,.
\end{align*}































