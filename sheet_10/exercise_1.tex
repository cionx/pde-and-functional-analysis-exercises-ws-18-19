\section{}

We suppose that~$f = \tilde{f}$ ought to be an equality in~$\Lp^p(U)$, and that~$f$ should be to be measurable.
We show that~$f_k \wto f$ for~$1 \leq p < \infty$, and that~$f_k \wtostar f$ for~$p = \infty$.
It then follows that~$f = \tilde{f}$ in~$\Lp^p(U)$ by the uniqueness of weak limits.

We first note that~$f \in \Lp^p(U)$:
For~$1 \leq p < \infty$ the sequence~$(f_k)_k$ is bounded in~$\Lp^p(U)$ because it is weakly convergent.
Hence
\[
  \norm{f}_p^p
  =
  \int_U \abs{f}^p
  =
  \int_U \liminf_{k \to \infty} \abs{f_k}^p
  \leq
  \liminf_{k \to \infty} \int_U \abs{f_k}^p
  =
  \liminf_{k \to \infty} \norm{f_k}_p^p
  <
  \infty
\]
by Fatou’s lemma.
For~$p = \infty$ we similarly find that the sequence~$(f_k)_k$ is bounded in~$\Lp^\infty(U)$ because it is {\weaklystar} convergent.
It follows that
\[
  \abs{f(x)}
  =
  \lim_{k \to \infty} \abs{f_k(x)}
  \leq
  \limsup_{k \to \infty} \norm{f_k}_\infty
  <
  \infty
\]
for almost all~$x \in U$, and hence that~$\norm{f}_\infty < \infty$.





\subsection*{The case $1 \leq p < \infty$}

We consider first the case~$1 \leq p < \infty$.
We fix~$T \in \Lp^p(U)'$ and show that~$T f_k \to T f$.
For this we may replace~$f_k$ by~$f_k - f$ to assume that~$f = 0$.
We hence need to show that~$T f_k \to 0$.
Let~$\varepsilon > 0$.

We know that~$\Lp^p(U)' = \Lp^q(U)$ with~$1/p + 1/q = 1$, in the sense that there exists some~$g \in \Lp^q(U)$ with
\[
    T h
  = \int_U g h
\]
for every~$h \in \Lp^p(U)$.
We therefore have for every~$k$ that
\[
  \abs{ T f_k }
  =
  \abs*{ \int_U g f_k }
  \leq
  \int_U \abs{g f_k} \,.
\]
We will split this integral up into three parts:

We first show that there exists a subset~$A \subseteq U$ of finite measure with~$\norm{g}_{q,U \setminus A} < \varepsilon$.
Indeed, there exists an increasing sequence
\[
  A_1
  \subseteq
  A_2
  \subseteq
  A_3
  \subseteq
  \dotsb
\]
of subsets~$A_n \subseteq U$ of finite measure with~$U = \bigcup_n A_n$ because~$U$ is~\dash{$\sigma$}{finite}.
It follows from Lebesgue’s monotone convergence theorem that
\[
  \norm{g}_{q,A_n}^q
  =
  \int_{A_n} \abs{g}^q
  =
  \int_U \chi_{A_n} \abs{g}^q
  \to
  \int_U \abs{g}^q
  =
  \norm{g}_{q,U}^q
\]
and hence~$\norm{g}_{q, U \setminus A_n} \to 0$.
For~$n$ sufficiently large we hence have~$\norm{g}_{q, U \setminus A_n} < \varepsilon$, and may choose~$A \defined A_n$.

The sequence~$(f_k)_k$ is bounded in~$\Lp^p(U)$, hence there exists~$C > 0$ with~$\norm{f_k}_{p,U} < C$ for every~$k$.
It follows for every~$k$ that
\begin{gather*}
  \int_U \abs{g f_k}
  =
  \int_{A} \abs{g f_k}
  +
  \int_{U \setminus A} \abs{g f_k}
\shortintertext{with}
  \int_{U \setminus A} \abs{g f_k}
  \leq
  \norm{g}_{q, U \setminus A} \norm{f_k}_{p, U \setminus A}
  \leq
  \norm{g}_{q, U \setminus A} \norm{f_k}_{p,U}
  \leq
  C \varepsilon \,.
\end{gather*}

We have that~$f_k \to 0$ almost everywhere, and hence there exists by Egorov’s theorem for every~$\delta > 0$ some measurable subset~$A' \subseteq A$ with~$\lambda(A \setminus A') < \delta$, such that~$f_k \to 0$ uniformly on~$A'$.
(Here~$\lambda$ denotes the Lebesgue measure.)
It follows that there exists a measurable subset~$A' \subseteq A$ with~$\norm{g}_{q,A \setminus A'} < \varepsilon$ such that~$f_k \to 0$ uniformly on~$A'$:
Indeed, there exists for every~$n \geq 1$ some measurable subset~$A'_n \subseteq A$ with~$\lambda(A \setminus A'_n) < 1/n$ such that~$f_k \to 0$ uniformly on~$A'_n$.
We may assume that~$A'_n \subseteq A_{n+1}$ for every~$n$ by replacing the set~$A'_n$ with~$\bigcup_{m \leq n} A_m$ (as given so far) for every~$n$.
Then
\[
  \lambda\left( U \setminus \bigcup_{n=1}^\infty A'_n \right)
  =
  \lambda\left( \bigcap_{n=1}^\infty (U \setminus A'_n) \right)
  =
  \lim_{n \to \infty} \lambda(U \setminus A'_n)
  =
  0
\]
and hence~$\chi_{A'_n} \to 1$ almost everywhere on~$A$.
It follows from another application of Lebesgue’s monotone convergence theorem that
\[
  \norm{g}_{q,A'_n}^q
  =
  \int_{A'_n} \abs{g}^q 
  =
  \int_A \chi_{A'_n} \abs{g}^q
  \to
  \int_A \abs{g}^q
  =
  \norm{g}_{q,A}^q \,,
\]
and therefore~$\norm{g}_{q,A \setminus A'_n} \to 0$.
For~$n$ sufficiently large we hence have~$\norm{g}_{q,A \setminus A'_n} < \varepsilon$, and may choose~$A' \defined A'_n$.
We now have with~$C'_{A'} \defined \norm{g}_{q,U} \lambda(A')^{1/p}$ that
\begin{align*}
  \int_A \abs{g f_k}
  &=
  \int_{A'} \abs{g f_k}
  +
  \int_{A \setminus A'} \abs{g f_k}
  \\
  &\leq
  \norm{g}_{q,A'} \norm{f_k}_{p,A'}
  +
  \norm{g}_{q, A \setminus A'} \norm{f_k}_{p, A \setminus A'}
  \\
  &\leq
  \norm{g}_{q,U} \norm{f_k}_{\infty,A'} \lambda(A')^{1/p}
  +
  \norm{g}_{q, A \setminus A'} \norm{f_k}_{p, U}
  \\
  &\leq
  C'_{A'} \norm{f_k}_{\infty, A'}
  +
  C \varepsilon \,.
\end{align*}
Altogether we have that
\[
  \abs{T f_k}
  \leq
  \int_U \abs{g f_k}
  =
  \int_{A'} \abs{g f_k}
  +
  \int_{A \setminus A'} \abs{g f_k}
  +
  \int_{U \setminus A} \abs{g f_k}
  \leq
  C'_{A'} \norm{f_k}_{\infty, A'} + 2 C \varepsilon \,.
\]
We have~$f_k \to 0$ uniformly on~$A'$ and hence~$\norm{f_k}_{\infty, A'} \to 0$.
We find that
\[
  \limsup_{k \to \infty} \abs{T f_k}
  \leq
  2 C \varepsilon \,.
\]
By letting~$\varepsilon \to 0$ we find that~$\abs{T f_k} \to 0$.





\subsection*{The case~$p = \infty$}

We fix~$g \in \Lp^1(U)$ and need to show that
\[
  \int_U f_k g
  \to
  \int_U f g \,.
\]
This follows from the same argumentation as above.





% We have for every subset~$A \subseteq U$ and every~$h \in \Lp^p(U)$ that
% \[
%   \abs{Th}
%   \leq
%   \int_U \abs{gh}
%   =
%   \int_A \abs{gh}
%   +
%   \int_{U \setminus A} \abs{gh} \,.
% \]
% We have for the first summand that
% \[
%   \int_A \abs{gh}
%   \leq
%   \norm{h}_{p,A}
%   \leq
%   \norm{g}_{q,A} \lambda(A)^{1/p} \norm{h}_{\infty,A}
% \]
% where~$\lambda$ denotes the Lebesgue measure.
% We have for the second sumand that
% \[
%   \int_{U \setminus A} \abs{gh}
%   \leq
%   \norm{h}_{p, U \setminus A} \norm{g}_{q, U \setminus A}
%   \leq
%   \norm{h}_{p, U} \norm{g}_{q, U \setminus A}
% \]
% We thus have that
% \[
%   \abs{ T f_k }
%   \leq
%   \norm{g}_{q,A} \lambda(A)^{1/p} \norm{f_k}_{\infty, A}
%   +
%   \norm{f_k}_{p,U} \norm{g}_{q, U \setminus A}
%   \leq
%   C_A \norm{f_k}_{\infty, A}
%   +
%   C'_{A'} \norm{g}_{q, U \setminus A}
% \]
% for~$C_A \defined \norm{g}_{q,U} \lambda(A)^{1/p}$ and~$C'_{A'} \defined \sup_k \norm{f_k}_{\Lp^p(U)} < \infty$.
% (Here we use that the sequence~$(f_k)_k$ is bounded in~$\Lp^p(U)$.)
% 
% The idea is now for~$\varepsilon > 0$ to choose~$A$ to be of finite measure but big enough so that~$\norm{g}_{q, U \setminus A} < \varepsilon$, and then to make~$k$ big enough so that~$\norm{f_k}_{\Lp^p(U)} < \varepsilon$:
% 
% There exists an increasing sequence~$(A_n)_n$ of subsets~$A_n \subseteq U$ with finite measure since~$U$ is~\dash{$\sigma$}{finite}.
% It follows from the Lebesgue’s monotone convergence theorem that
% \[
%   \norm{g}_{q,A_n}
%   \to
%   \norm{g}_{q,U} \,,
% \]
% and hence that~$\norm{g}_{q, U \setminus A_n} \to 0$.
% For~$n$ sufficiently large we therefore get~$\norm{g}_{q, U \setminus A_n} < \varepsilon$, and we choose~$\tilde{A} \defined A_n$.
% 
% The subset~$\tilde{A} \subseteq X$ has finite measure, and~$f_k \to f$ almost everywhere.
% It follows from 




