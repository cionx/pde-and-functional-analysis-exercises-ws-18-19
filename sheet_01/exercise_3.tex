\section{}





\addtocounter{subsection}{1}





\subsection{}


\begin{lemma}
  \label{complete iff closed}
  Let~$(X,d)$ be a complete metric space.
  Then a subspace~$A$ of~$X$ is closed if and only if it again complete.
\end{lemma}


\begin{proof}
  If~$A$ is closed then every Cauchy sequence in~$A$ has a limit in~$X$, which is then again contained in~$A$.
  If~$A$ is complete and~$x \in X$ is the limit of a sequence~$(a_n)_n \subseteq A$, then the sequence~$(a_n)_n$ is in particular Cauchy and therefore has a limit in~$A$, which must coincide with~$x$ by the uniqueness of limits (in metric spaces).
\end{proof}


It sufficies by~\cref{complete iff closed} to show that~$\ell_1$ is not closed in~$\ell_\infty$ with respect to the norm~$\norm{\,\cdot\,}_\infty$.
For this we consider the sequence~$x \in \ell_\infty$ with~$x_n = 1/n$.
The sequence~$x$ is not contained in~$\ell_1$
But it is with respect to~$\norm{\,\cdot\,}$ the limit of the sequence~$x^{(k)} \in \ell_1$ given by
\[
              x^{(k)}_n
    \defined  \begin{cases}
                1/n & \text{if~$n \leq k$} \,,  \\
                0   & \text{else} \,,
              \end{cases}
\]
because
\[
          \norm*{x - x^{(k)}}_\infty
  =       \sum_{n > k} \frac{1}{n}
  =       \frac{1}{k+1}
  \longto 0
\]
for~$k \to \infty$.
This shows that~$\ell_1$ is with respect to the norm~$\norm{\,\cdot\,}$ not closed in~$\ell_\infty$, and therefore not complete with respect to~$\norm{\,\cdot\,}$.
