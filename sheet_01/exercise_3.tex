\section{}





\subsection{}

Let~$(x^{(k)})_k$ be a Cauchy sequence in~$(\ell_p, \norm{\,\cdot\,}$.
It then holds for every~$n \geq 1$ that the projection~$\pi_n \colon \ell_p \to \Real$,~$x \mapsto x_n$ is {\lipcont} with {\lipconst}~$1$, from which it follows that~$(x^{(k)}_n)_k$ is again a Cauchy sequence, and thus convergent.
For every~$n \geq 1$ let~$x_n \defined \lim_{k \to \infty} x^{(k)}_n$.

The sequence~$x$ is again contained in~$\ell_p$ since it follows from Fatou’s~lemma that
\[
        \norm{x}_p^p
  =     \sum_{n=1}^\infty \abs{x_n}^p
  =     \sum_{n=1}^\infty \lim_{k \to \infty} \abs{x_n^{(k)}}^p
  \leq  \liminf_{k \to \infty} \sum_{n=1}^\infty \abs{x_n^{(k)}}^p
  =     \liminf_{k \to \infty} \norm{x^{(k)}}
  <     \infty \,,
\]
because the sequence~$(x^{(k)})_k$ is bounded.

To show that the Cauchy sequence~$(x^{(k)})_k$ converges to~$x$ with respect to the norm~$\norm{\,\cdot\,}_p$ we may replace this sequence by any of its subsequences.
We may therefore assume that~$\norm{x^{(l)} - x^{(k)}} \leq 1/2^k$ for all~$l \geq k$.
It then follows from another use of Fatou’s~lemma that
\begin{align*}
        \norm{x - x^{(k)}}_p^p
  &=    \sum_{n=1}^\infty \abs{ x_n - x^{(k)}_n }^p
   =    \sum_{n=1}^\infty \lim_{l \to \infty} \abs{ x^{(l)}_n - x^{(k)}_n }^p \\
  &\leq \liminf_{l \to \infty} \sum_{n=1}^\infty \abs{ x^{(l)}_n - x^{(k)}_n }^p
   =    \liminf_{l \to \infty} \norm{x^{(l)} - x^{(k)}}
   \leq \liminf_{l \to \infty} \frac{1}{2^k}
   =    \frac{1}{2^k} \,,
\end{align*}
which shows that~$x^{(k)} \to x$.
Altogether this shows that the sequence~$(x^{(k)})_k$ converges in~$(\ell_p, \norm{\,\cdot\,})$.






\subsection{}


\begin{lemma}
  \label{complete iff closed}
  Let~$(X,d)$ be a complete metric space.
  Then a subspace~$A$ of~$X$ is closed if and only if it again complete.
\end{lemma}


\begin{proof}
  If~$A$ is closed then every Cauchy sequence in~$A$ has a limit in~$X$, which is then again contained in~$A$.
  If~$A$ is complete and~$x \in X$ is the limit of a sequence~$(a_n)_n \subseteq A$, then the sequence~$(a_n)_n$ is in particular Cauchy and therefore has a limit in~$A$, which must coincide with~$x$ by the uniqueness of limits (in metric spaces).
\end{proof}


It sufficies by~\cref{complete iff closed} to show that~$\ell_1$ is not closed in~$\ell_\infty$ with respect to the norm~$\norm{\,\cdot\,}_\infty$.
For this we consider the sequence~$x \in \ell_\infty$ with~$x_n = 1/n$.
The sequence~$x$ is not contained in~$\ell_1$
But it is with respect to~$\norm{\,\cdot\,}$ the limit of the sequence~$x^{(k)} \in \ell_1$ given by
\[
              x^{(k)}_n
    \defined  \begin{cases}
                1/n & \text{if~$n \leq k$} \,,  \\
                0   & \text{else} \,,
              \end{cases}
\]
because
\[
          \norm*{x - x^{(k)}}_\infty
  =       \sum_{n > k} \frac{1}{n}
  =       \frac{1}{k+1}
  \longto 0
\]
for~$k \to \infty$.
This shows that~$\ell_1$ is with respect to the norm~$\norm{\,\cdot\,}$ not closed in~$\ell_\infty$, and therefore not complete with respect to~$\norm{\,\cdot\,}$.
