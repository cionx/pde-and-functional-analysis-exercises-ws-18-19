\section{}





\subsection*{Interior of~$A_1$}

The interior of~$A_1$ is empty:
Let~$x \in A_1$ and let~$\varepsilon > 0$.
It follows from~$x$ being a positive null sequence that there exists some~$N \in \Natural$ with~$0 < x_N < \varepsilon/2$.
The sequence~$y$ with
\[
            y_n
  \defined  \begin{cases}
              x_n - \varepsilon & \text{if~$n = N$} \,, \\
              x_n               & \text{else} \,, \\
            \end{cases}
\]
differs with~$x$ only by a single value, and is therefore again in~$\ell_1$.
It holds that~$y_N < -\varepsilon/2$ and therefore that~$y_N \notin A_1$, but
\[
    \norm{x - y}_1
  = \varepsilon \,.
\]
This shows that~$A_1$ does not contain any open ball around~$x$.





\subsection*{Closure of~$A_1$}

The closure of~$A_1$ is given by the set
\[
            C
  \defined  \{
              x \in \ell_1
            \suchthat
              \text{$x_n \geq 0$ for all~$n \in \Natural$}
            \} \,.
\]
To show that~$\closure{A_1} \subseteq C$ we note that the projection map
\[
          \pi_n
  \colon  \ell_1
  \to     \Real \,,
  \quad   x
  \mapsto x_n
\]
is~{\lipcont} with {\lipconst}~$1$ for every~$n \geq 1$, hence continuous, and that therefore
\[
            \pi_n(\closure{A_1})
  \subseteq \closure{ \pi_n(A_1) }
  =         \closure{ (0,\infty) }
  =         [0,\infty) \,.
\]
To show the inclusion~$C \subseteq \closure{A_1}$ we define for~$x \in C$ and~$\varepsilon > 0$ the sequence~$y$ with
\[
            y_n
  \defined  \begin{cases}
              x_n             & \text{if~$x_n > 0$} \,,  \\
              \varepsilon/2^n & \text{otherwise} \,.
            \end{cases}
\]
This sequence is again in~$\ell_1$, because both~$x$ and~$(1/2^n)_{n \geq 1}$ are so, and it is even contained in~$A_1$.
It holds that
\[
        \norm{ x - y }_1
  =     \norm{ (x_n - y_n)_n }
  =     \sum_{\substack{n \geq 1 \\ x_n = 0}} \frac{\varepsilon}{2^n}
  \leq  \sum_{n=1}^\infty \frac{\varepsilon}{2^n}
  =     \varepsilon \,,
\]
as desired.





\subsection*{Interior of~$A_2$}

The interior of~$A_2$ is empty:
If~$x \in A_2$ and~$\varepsilon > 0$ then the sequence~$y$ with
\[
            y_n
  \defined  \begin{cases}
              x_n             & \text{if~$x_n \neq 0$} \,,  \\
              \varepsilon/2^n & \text{otherwise} \,,
            \end{cases}
\]
is again in~$\ell_1$ with~$\norm{ x - y }_1 \leq \varepsilon$, as seen above.
But the sequence~$y$ is not contained in~$A_2$, since~$y_n \neq 0$ for every~$n$.
This shows that~$A_2$ does not contain an open ball around~$x$.





\subsection*{Closure of~$A_2$}

The set~$A_2$ is dense in~$\ell_1$:
If~$(x_n)_n \in \ell_1$ then we can define the sequence~$(x^{(k)})_k$ in~$A_1$ by cutting off the sequence~$(x_n)_n$ after~$k$ terms:
\[
            x^{(k)}_n
  \defined  \begin{cases}
              x_n & \text{if $n \leq k$} \,,  \\
              0   & \text{else} \,.
            \end{cases}
\]
It then holds that
\[
          \norm*{ x - x^{(k)} }_1
  =       \norm*{ (x_n - x^{(k)}_n)_n }_1
  =       \sum_{n={k+1}}^\infty \norm{x_n}
  \longto  0
\]
for~$k \to \infty$ because the series~$\sum_{n=1}^\infty \norm{x_n}$ converges.

This shows that every sequences~$x \in \ell_1$ can be approximated by a sequence~$(x^{(k)})_k$ in~$A_2$, which shows that~$A_2$ is dense in~$\ell_1$, and therefore that~$\closure{A_2} = \ell_1$.




